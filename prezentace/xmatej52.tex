\documentclass{beamer}
\usepackage{times}
\usepackage[czech]{babel}
\usepackage[utf8]{inputenc}
% \usetheme{Berkeley}
% \usetheme{Copenhagen}
\usetheme{Madrid}
% \usecolortheme{beaver}
\newcommand{\czuv}[1]{\quotedblbase #1\textquotedblleft}
\usepackage{graphicx}
\usepackage{amsmath}
\setbeamertemplate{caption}[numbered]

\title{Rozšíření systému pro získávání, zpracování a analýzu rozsáhlých kolekcí textů z webu}
\subtitle{Prezentace k předmětu Semestální projekt}
\author{Matějka Jiří}
\date{\today}
\logo{\includegraphics[width=3cm]{logo.png}}

\begin{document}
  \frame{\titlepage}
%%%%%%%%%%%%%%%%%%%%%%%%%%%%%%%%%%%%%%%%%%%%
  \begin{frame}
    \frametitle{Obsah}

    \begin{itemize}
      \item Linux Mint
      \item Linux Mint - osobní zkušenost
      \item Linux Mint Debian Edition
      \item Linux Mint Debian Edition - osobní zkušenost
      \item Ubuntu
      \item Ubuntu - osobní zkušenost
      \item Debian
      \item Debian - osobní zkušenost
    \end{itemize}
  \end{frame}
%%%%%%%%%%%%%%%%%%%%%%%%%%%%%%%%%%%%%%%%%%%%
  \begin{frame}
    \frametitle{Ubuntu}
    \hspace{2mm}
    \includegraphics[height=1.5cm]{ubuntu.png}
    \footnotetext[1]{Zdroj: Ubuntu logo [online] [vid. 2016-04-06], \\ \hspace{5.5mm}dostupné z: http://design.ubuntu.com/brand/ubuntu-logo \vspace{1.5mm}}
	\begin{itemize}
		\item Nejnovější verze: 16.04 LTS
		\item GUI: Unity, GNOME, KDE, XFCE, LXDE, ...
		\item Kernel: 4.4
		\item Výhody: Početná komunita, snadná instalace, podpora hardware, LTS (Long Term Support) verze, nejnovější LTS verze jádra
		\item Nevýhody: Angažování firmy do vývoje OS, možná instalace nestabilních balíků
	\end{itemize}
  \end{frame}
%%%%%%%%%%%%%%%%%%%%%%%%%%%%%%%%%%%%%%%%%%%%
  \begin{frame}
    \frametitle{Ubuntu - Vlastní zkušenost}
      \begin{itemize}
	\item Verze: 15.10
	\item GUI: GNOME
	\item Kernel: 4.2
      \end{itemize}
      Pro Ubuntu jsem se rozhodl poté, co jsem se pokusil ručně aktualizovat
      jádro systému Linux Mint Debian Edition. Distribuce mi vůbec nevyhovovala,
      co chvíli se objevilo vyskakovací okno s hlášením nějaké chyby. Žádný operační systém
      nebyl odstraněn z mého zařízení tak rychle, jako tato verze Ubuntu.
  \end{frame}
%%%%%%%%%%%%%%%%%%%%%%%%%%%%%%%%%%%%%%%%%%%%%
  \begin{frame}
    \frametitle{Linux Mint}
    \includegraphics[height=1.5cm]{mint.png}
    \footnotetext[1]{Zdroj: Linux Mint Official Logo.svg [online] [vid. 2016-04-06], \\ \hspace{5.5mm}dostupné z: https://commons.wikimedia.org/wiki/File:Linux\_Mint\_Official\_Logo.svg \vspace{1.5mm}}
      \begin{itemize}
	\item Nejnovější verze: 17.3 \czuv{Rosa}
	\item GUI: KDE, Cinnamon, Mate a XFCE
	\item Kernel: 3.16
	\item Výhody: Početná komunita, snadná instalace, podpora hardware, malé HW požadavky
	\item Nevýhody: Možná instalace nestabilních balíků, starší verze jádra
      \end{itemize}
  \end{frame}
%%%%%%%%%%%%%%%%%%%%%%%%%%%%%%%%%%%%%%%%%%%%%%%%
  \begin{frame}
    \frametitle{Linux Mint - Vlastní zkušenost}
      \begin{itemize}
	\item Verze: 15, 16, 17, 17.1 a 17.2
	\item GUI: Cinnamon, MATE a KDE
	\item Kernel: 3.16
      \end{itemize}
      Linux Mint byla první distribuce Linuxu, kterou jsem vyzkoušel. Zpočátku jsem ji moc nepoužíval,
      ale poté, co úplně selhal Windows~Vista a následně i~Windows~7, jsem přešel k
      Linuxu. Linux Mint je jedna z nejvhodnějších distribucí pro začátečníky a
      je do jisté míry i podobná Windows. S~distribucí jsem byl naprosto spokojený,
      byť grafické prostředí KDE způsobovalo mému zastaralému stroji značné problémy.
      Po přechodu na Cinnamon veškeré problémy ustaly.
  \end{frame}
%%%%%%%%%%%%%%%%%%%%%%%%%%%%%%%%%%%%%%%%%%%%%%%%%%%
  \begin{frame}
    \frametitle{Linux Mint Debian Edition}
    \includegraphics[height=1.5cm]{mint.png}
    \footnotetext[1]{Zdroj: Linux Mint Official Logo.svg [online] [vid. 2016-04-06], \\ \hspace{5.5mm}dostupné z: https://commons.wikimedia.org/wiki/File:Linux\_Mint\_Official\_Logo.svg \vspace{1.5mm}}
      \begin{itemize}
	\item Nejnovější verze: 2.0 \czuv{Betsy}
	\item GUI: Cinnamon a Mate
	\item Kernel: 3.16
	\item Výhody: Početná komunita, snadná instalace, podpora hardware, malé HW požadavky, stabilita, stabilní instalační balíky
	\item Nevýhody: Starší verze jádra, pomalejší vývoj OS
      \end{itemize}
  \end{frame}
%%%%%%%%%%%%%%%%%%%%%%%%%%%%%%%%%%%%%%%%%%%%%%%%%%%%%%
  \begin{frame}
    \frametitle{Linux Mint Debian Edition - Vlastní zkušenost}
      \begin{itemize}
	\item Verze: 2.0 \czuv{Betsy}
	\item GUI: Cinnamon
	\item Kernel: 3.16
      \end{itemize}
      Tato distribuce se mi zatím nejvíce osvědčila. Ze všech distribucí, co jsem zkoušel,
      je nejstabilnější, ale má i své \czuv{mouchy}. Jádro 3.16 patří mezi starší verze
      a právě kvůli chybě v jeho kódu na mém laptopu nefunguje optimálně větráček. Upgrade na vyšší verzi
      jádra se mi nepovedl a~musel jsem přeinstalovávat celý operační systém, nicméně LMDE
      mi nejvíce \czuv{sedla} a momentálně tuto distribuci používám.
  \end{frame}
%%%%%%%%%%%%%%%%%%%%%%%%%%%%%%%%%%%%%%%%%%%%%%%%%%%%%%%%%%
  \begin{frame}
    \frametitle{Debian}
    \includegraphics[height=1.5cm]{Debian.jpg}
        \footnotetext[1]{Zdroj: Debian 7.1 [online] [vid. 2016-04-06], \\ \hspace{5.5mm}dostupné z: http://www.linux.org/resources/debian.11/ \vspace{1.5mm}}
      \begin{itemize}
	\item Nejnovější verze: 8.4 \czuv{Jessie}
	\item GUI: Cinnamon, KDE, LXDE, XFCE, ...
	\item Kernel: 3.16
	\item Výhody: Početná komunita, podpora hardware, malé HW požadavky, stabilita, stabilní instalační balíky
	\item Nevýhody: Starší verze jádra, pomalejší vývoj OS, je určený pro pokročilejší uživatele
      \end{itemize}
  \end{frame}
%%%%%%%%%%%%%%%%%%%%%%%%%%%%%%%%%%%%%%%%%%%%%%%%%%%%%%%%%%%
  \begin{frame}
    \frametitle{Debian - Vlastní zkušenost}
      \begin{itemize}
	\item Verze: 8.4 \czuv{Jessie}
	\item GUI: GNOME
	\item Kernel: 3.16
      \end{itemize}
      Pro Debian jsem se rozhodl po dobré zkušenosti s Linux Mint Debian Edition a nezklamal.
      Operační systém byl stabilní a oproti LMDE s~grafickým prostředím Cinnamon i o něco rychlejší,
      ale kvůli stejné verzi jádra, jako u LMDE a také kvůli tomu, že mě Minty \czuv{rozmazlily}
      jednoduchostí a lehkou ovladatelností, jsem se opět vrátil k LMDE.
  \end{frame}
%%%%%%%%%%%%%%%%%%%%%%%%%%%%%%%%%%%%%%%%%%%%%%%%%%%%%%%%%%%%%%
  \begin{frame}
    \frametitle{Závěr}
    \begin{center}
      Děkuji za pozornost
    \end{center}
  \end{frame}
\end{document}
